\documentclass[twoside]{article}
\title{Writing and Identifying Balanced Chemical Equations for Real Life Reactions}
\author{Timmy Rettelle and Jason Yundt}
\date{05/3/17}

\usepackage{enumitem}
\usepackage[version=4]{mhchem}
\usepackage[letterpaper]{geometry}

\begin{document}
   \maketitle
   \textbf{Purpose}: To examine chemical reactions and to determine the relevant chemical equations.
   \begin{enumerate}[label=\Roman*.]
    
    \item % I
    \begin{enumerate}[label=\arabic*.]
     \item Methane gas and oxygen combust forming water vapor and carbon dioxide gas.
     \item \ce{CH4(g) + 2O2(g) ->[sparker] 2H2O(g) + CO2(g)}
     \item Combustion
     \item N/A
    \end{enumerate}
    
    \item % II
    \begin{enumerate}[label=\arabic*.]
     \item Magnesium strips and oxygen gas are heated up and synthisize into a magnesium oxide powder.
     \item \ce{2Mg(s) + O2(g) ->[heat] 2MgO(g)}
     \item Synthisis
     \item N/A
    \end{enumerate}
    
    \item % III
    \begin{enumerate}[label=\arabic*.]
     \item A copper tube is put into a test tube filled with aqueous silver nitrate and after single replacement forms aqueous copper (II) nitrate and a silver percipitate.
     \item \ce{Cu(s) + 2AgNO3(aq) -> Cu(NO3)2(aq) + 2Ag(s)}
     \item Single replacement
     \item Total Ionic Equation: \\ \ce{Cu(s) + 2Ag^+(aq) + 2NO3^-(aq) -> Cu2^+(aq) + 2NO3^-(aq) + 2Ag(s)} \\
	   \\
	   Net Ionic Equation: \\ \ce{Cu(s) + 2Ag^+(aq) -> Cu^2+(aq) + 2Ag(s)}
    \end{enumerate}
    
    \item % IV
    \begin{enumerate}[label=\arabic*.]
     \item Liquid hydrogen peroxide with solid manganese dioxide as a catylist decomposes into liquid water and oxygen gas.
     \item \ce{2H2O2(l) ->[MnO2(s)] 2H2O(l) + O2(g)}
     \item Decomposition
     \item N/A
    \end{enumerate}
    
    \item % V
    \begin{enumerate}[label=\arabic*.]
     \item Solid zinc and aqueous hydrochloric acid under go single replacement, forming aquous zinc chloride and hydrogen gas.
     \item \ce{Zn(s) + 2HCl(aq) -> ZnCl2(aq) + H2(g)}
     \item Single replacement
     \item Total Ionic Equation: \\ \ce{Zn(s) + 2H^+(aq) + 2Cl^-(aq) -> Zn^2+(aq) + 2Cl^-(aq) + H2(g)} \\
	   \\
	   Net Ionic Equation: \\ \ce{Zn(s) + 2H^+(aq) -> Zn^2+(aq) + H2(g)}
    \end{enumerate}
    
    \item % VI
    \begin{enumerate}[label=\arabic*.]
     \item A copper (II) sulfate solution and a sodium hydroxide solution undergo double replacement, forming solid copper (II) hydroxide and aqueous sodium sulfate.
     \item \ce{CuSO4(aq) + 2NaOH(aq) -> Cu(OH)2(s) + Na2SO4(aq)}
     \item Double replacement
     \item Total Ionic Equation: \\ \ce{Cu^2+(aq) + SO4^2-(aq) + 2Na^+(aq) + 2OH^-(aq) -> Cu(OH)2(s) + 2Na^+(aq) + SO4^2-(aq)} \\
	   \\
	   Net Ionic Equation: \\ \ce{Cu^2+(aq) + 2OH^-(aq) -> Cu(OH)2(s)}
    \end{enumerate}
   
   \end{enumerate}
   
   \noindent \textbf{Conclusion}:
   
   A synthesis reaction is a reaction in which multiple less complex compounds react to form fewer, more complex compounds. A real life example of
	this is photosynthesis, when a plant takes in water and carbon dioxide to create food (glucose). The formula is \ce{6CO2 + 6H20 -> C6H12O + 6O2}.
	A decomposition reaction is the opposite of a synthesis. It is when more complex compounds break down to form many less complex compounds. An example
	of this is cellular respiration, which is how our bodies use glucose for energy. The formula is \ce{C6H12O6 + 6O2 -> 6CO2 + 6H2O}. 
	A single replacement reaction is when an ion that is alone switches with an ion that is in a compound. A real life example of this is the rusting of steel.
	The formula is \ce{Fe2 + 2H2O -> Fe(OH)2 + H2}. A double replacement reaction is when ions in 2 compounds switch places to form 2 different compounds.
	A real life example of this is antacids neutralizing stomach acids. The formula is \ce{Ca(OH) 2 + 2HCl -> CaCl2 + 2H2O}. A combustion reaction is when a hydrocarbons
	reacts with O2 and forms CO2 and water. A real life example is burning gasoline. The formula is \ce{2C8H18 + 25O2 -> 16CO2 + 18H2O}.


\end{document}